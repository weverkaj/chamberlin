\documentclass[]{article}
\usepackage{lmodern}
\usepackage{amssymb,amsmath}
\usepackage{ifxetex,ifluatex}
\usepackage{fixltx2e} % provides \textsubscript
\ifnum 0\ifxetex 1\fi\ifluatex 1\fi=0 % if pdftex
  \usepackage[T1]{fontenc}
  \usepackage[utf8]{inputenc}
\else % if luatex or xelatex
  \ifxetex
    \usepackage{mathspec}
  \else
    \usepackage{fontspec}
  \fi
  \defaultfontfeatures{Ligatures=TeX,Scale=MatchLowercase}
\fi
% use upquote if available, for straight quotes in verbatim environments
\IfFileExists{upquote.sty}{\usepackage{upquote}}{}
% use microtype if available
\IfFileExists{microtype.sty}{%
\usepackage{microtype}
\UseMicrotypeSet[protrusion]{basicmath} % disable protrusion for tt fonts
}{}
\usepackage[margin=1in]{geometry}
\usepackage{hyperref}
\hypersetup{unicode=true,
            pdftitle={Chamberlin CO2 Emission Analysis},
            pdfborder={0 0 0},
            breaklinks=true}
\urlstyle{same}  % don't use monospace font for urls
\usepackage{color}
\usepackage{fancyvrb}
\newcommand{\VerbBar}{|}
\newcommand{\VERB}{\Verb[commandchars=\\\{\}]}
\DefineVerbatimEnvironment{Highlighting}{Verbatim}{commandchars=\\\{\}}
% Add ',fontsize=\small' for more characters per line
\usepackage{framed}
\definecolor{shadecolor}{RGB}{248,248,248}
\newenvironment{Shaded}{\begin{snugshade}}{\end{snugshade}}
\newcommand{\KeywordTok}[1]{\textcolor[rgb]{0.13,0.29,0.53}{\textbf{#1}}}
\newcommand{\DataTypeTok}[1]{\textcolor[rgb]{0.13,0.29,0.53}{#1}}
\newcommand{\DecValTok}[1]{\textcolor[rgb]{0.00,0.00,0.81}{#1}}
\newcommand{\BaseNTok}[1]{\textcolor[rgb]{0.00,0.00,0.81}{#1}}
\newcommand{\FloatTok}[1]{\textcolor[rgb]{0.00,0.00,0.81}{#1}}
\newcommand{\ConstantTok}[1]{\textcolor[rgb]{0.00,0.00,0.00}{#1}}
\newcommand{\CharTok}[1]{\textcolor[rgb]{0.31,0.60,0.02}{#1}}
\newcommand{\SpecialCharTok}[1]{\textcolor[rgb]{0.00,0.00,0.00}{#1}}
\newcommand{\StringTok}[1]{\textcolor[rgb]{0.31,0.60,0.02}{#1}}
\newcommand{\VerbatimStringTok}[1]{\textcolor[rgb]{0.31,0.60,0.02}{#1}}
\newcommand{\SpecialStringTok}[1]{\textcolor[rgb]{0.31,0.60,0.02}{#1}}
\newcommand{\ImportTok}[1]{#1}
\newcommand{\CommentTok}[1]{\textcolor[rgb]{0.56,0.35,0.01}{\textit{#1}}}
\newcommand{\DocumentationTok}[1]{\textcolor[rgb]{0.56,0.35,0.01}{\textbf{\textit{#1}}}}
\newcommand{\AnnotationTok}[1]{\textcolor[rgb]{0.56,0.35,0.01}{\textbf{\textit{#1}}}}
\newcommand{\CommentVarTok}[1]{\textcolor[rgb]{0.56,0.35,0.01}{\textbf{\textit{#1}}}}
\newcommand{\OtherTok}[1]{\textcolor[rgb]{0.56,0.35,0.01}{#1}}
\newcommand{\FunctionTok}[1]{\textcolor[rgb]{0.00,0.00,0.00}{#1}}
\newcommand{\VariableTok}[1]{\textcolor[rgb]{0.00,0.00,0.00}{#1}}
\newcommand{\ControlFlowTok}[1]{\textcolor[rgb]{0.13,0.29,0.53}{\textbf{#1}}}
\newcommand{\OperatorTok}[1]{\textcolor[rgb]{0.81,0.36,0.00}{\textbf{#1}}}
\newcommand{\BuiltInTok}[1]{#1}
\newcommand{\ExtensionTok}[1]{#1}
\newcommand{\PreprocessorTok}[1]{\textcolor[rgb]{0.56,0.35,0.01}{\textit{#1}}}
\newcommand{\AttributeTok}[1]{\textcolor[rgb]{0.77,0.63,0.00}{#1}}
\newcommand{\RegionMarkerTok}[1]{#1}
\newcommand{\InformationTok}[1]{\textcolor[rgb]{0.56,0.35,0.01}{\textbf{\textit{#1}}}}
\newcommand{\WarningTok}[1]{\textcolor[rgb]{0.56,0.35,0.01}{\textbf{\textit{#1}}}}
\newcommand{\AlertTok}[1]{\textcolor[rgb]{0.94,0.16,0.16}{#1}}
\newcommand{\ErrorTok}[1]{\textcolor[rgb]{0.64,0.00,0.00}{\textbf{#1}}}
\newcommand{\NormalTok}[1]{#1}
\usepackage{graphicx,grffile}
\makeatletter
\def\maxwidth{\ifdim\Gin@nat@width>\linewidth\linewidth\else\Gin@nat@width\fi}
\def\maxheight{\ifdim\Gin@nat@height>\textheight\textheight\else\Gin@nat@height\fi}
\makeatother
% Scale images if necessary, so that they will not overflow the page
% margins by default, and it is still possible to overwrite the defaults
% using explicit options in \includegraphics[width, height, ...]{}
\setkeys{Gin}{width=\maxwidth,height=\maxheight,keepaspectratio}
\IfFileExists{parskip.sty}{%
\usepackage{parskip}
}{% else
\setlength{\parindent}{0pt}
\setlength{\parskip}{6pt plus 2pt minus 1pt}
}
\setlength{\emergencystretch}{3em}  % prevent overfull lines
\providecommand{\tightlist}{%
  \setlength{\itemsep}{0pt}\setlength{\parskip}{0pt}}
\setcounter{secnumdepth}{0}
% Redefines (sub)paragraphs to behave more like sections
\ifx\paragraph\undefined\else
\let\oldparagraph\paragraph
\renewcommand{\paragraph}[1]{\oldparagraph{#1}\mbox{}}
\fi
\ifx\subparagraph\undefined\else
\let\oldsubparagraph\subparagraph
\renewcommand{\subparagraph}[1]{\oldsubparagraph{#1}\mbox{}}
\fi

%%% Use protect on footnotes to avoid problems with footnotes in titles
\let\rmarkdownfootnote\footnote%
\def\footnote{\protect\rmarkdownfootnote}

%%% Change title format to be more compact
\usepackage{titling}

% Create subtitle command for use in maketitle
\providecommand{\subtitle}[1]{
  \posttitle{
    \begin{center}\large#1\end{center}
    }
}

\setlength{\droptitle}{-2em}

  \title{Chamberlin CO2 Emission Analysis}
    \pretitle{\vspace{\droptitle}\centering\huge}
  \posttitle{\par}
    \author{}
    \preauthor{}\postauthor{}
    \date{}
    \predate{}\postdate{}
  

\begin{document}
\maketitle

Introducting packages, including gasanalyzR, my own personal set of
functions that I'm storing for this type of analysis. I can provide more
detail in how this works if necessary.

\begin{Shaded}
\begin{Highlighting}[]
\KeywordTok{library}\NormalTok{(tidyverse)}
\end{Highlighting}
\end{Shaded}

\begin{verbatim}
## Warning: package 'tidyverse' was built under R version 3.5.3
\end{verbatim}

\begin{verbatim}
## -- Attaching packages -------------------------------------------------------------------------------------------------------------------------- tidyverse 1.3.0 --
\end{verbatim}

\begin{verbatim}
## v ggplot2 3.2.1     v purrr   0.3.3
## v tibble  2.1.3     v dplyr   0.8.4
## v tidyr   1.0.2     v stringr 1.4.0
## v readr   1.3.1     v forcats 0.4.0
\end{verbatim}

\begin{verbatim}
## Warning: package 'ggplot2' was built under R version 3.5.3
\end{verbatim}

\begin{verbatim}
## Warning: package 'tibble' was built under R version 3.5.3
\end{verbatim}

\begin{verbatim}
## Warning: package 'tidyr' was built under R version 3.5.3
\end{verbatim}

\begin{verbatim}
## Warning: package 'readr' was built under R version 3.5.3
\end{verbatim}

\begin{verbatim}
## Warning: package 'purrr' was built under R version 3.5.3
\end{verbatim}

\begin{verbatim}
## Warning: package 'dplyr' was built under R version 3.5.3
\end{verbatim}

\begin{verbatim}
## Warning: package 'stringr' was built under R version 3.5.3
\end{verbatim}

\begin{verbatim}
## Warning: package 'forcats' was built under R version 3.5.3
\end{verbatim}

\begin{verbatim}
## -- Conflicts ----------------------------------------------------------------------------------------------------------------------------- tidyverse_conflicts() --
## x dplyr::filter() masks stats::filter()
## x dplyr::lag()    masks stats::lag()
\end{verbatim}

\begin{Shaded}
\begin{Highlighting}[]
\KeywordTok{library}\NormalTok{(broom)}
\end{Highlighting}
\end{Shaded}

\begin{verbatim}
## Warning: package 'broom' was built under R version 3.5.3
\end{verbatim}

\begin{Shaded}
\begin{Highlighting}[]
\KeywordTok{library}\NormalTok{(gasanalyzR)}
\KeywordTok{library}\NormalTok{(colorspace)}
\end{Highlighting}
\end{Shaded}

\begin{verbatim}
## Warning: package 'colorspace' was built under R version 3.5.3
\end{verbatim}

The data - including some preliminatry wrangling including dropping NAs,
selecting only columns that we'll use. Daily highs temperatures are from
PRISM.

\begin{Shaded}
\begin{Highlighting}[]
\NormalTok{CO2_measurements =}\StringTok{ }\KeywordTok{read.csv}\NormalTok{(}\StringTok{"../data/csv/CO2.csv"}\NormalTok{)}
\NormalTok{  CO2_measurements}\OperatorTok{$}\NormalTok{peak.heights..um.m. =}\StringTok{ }\KeywordTok{as.numeric}\NormalTok{(}\KeywordTok{as.character}\NormalTok{(CO2_measurements}\OperatorTok{$}\NormalTok{peak.heights..um.m.))}
\NormalTok{co2_standards_mV =}\StringTok{ }\KeywordTok{read.csv}\NormalTok{(}\StringTok{"../data/csv/CO2_standards.csv"}\NormalTok{) }\OperatorTok\StringTok{ }\KeywordTok{select}\NormalTok{(date, std.ppm, co2mV) }\OperatorTok\StringTok{ }\KeywordTok{drop_na}\NormalTok{()}
\NormalTok{co2_standards_mmol =}\StringTok{ }\KeywordTok{read.csv}\NormalTok{(}\StringTok{"../data/csv/CO2_standards.csv"}\NormalTok{) }\OperatorTok\StringTok{ }\KeywordTok{select}\NormalTok{(date, std.ppm, co2mmol.mol) }\OperatorTok\StringTok{ }\KeywordTok{drop_na}\NormalTok{()}
\NormalTok{buckets =}\StringTok{ }\KeywordTok{read.csv}\NormalTok{(}\StringTok{"../data/csv/buckets.csv"}\NormalTok{)}
\NormalTok{dailyhighs =}\StringTok{ }\KeywordTok{read.csv}\NormalTok{(}\StringTok{"../data/csv/dailyhighs.csv"}\NormalTok{)}
\end{Highlighting}
\end{Shaded}

Create standard equations using standard.curve() from gasanalyzR

\begin{Shaded}
\begin{Highlighting}[]
\NormalTok{standard_equations_mV =}\StringTok{ }\KeywordTok{standard.curve}\NormalTok{(co2_standards_mV, }\DataTypeTok{peak_field =} \StringTok{"co2mV"}\NormalTok{, }\DataTypeTok{std_field =} \StringTok{"std.ppm"}\NormalTok{)}
\NormalTok{standard_equations_mmol =}\StringTok{ }\KeywordTok{standard.curve}\NormalTok{(co2_standards_mmol, }\DataTypeTok{peak_field =} \StringTok{"co2mmol.mol"}\NormalTok{, }\DataTypeTok{std_field =} \StringTok{"std.ppm"}\NormalTok{)}
\end{Highlighting}
\end{Shaded}

Wrangling daily high data

\begin{Shaded}
\begin{Highlighting}[]
\NormalTok{high =}\StringTok{ }\NormalTok{dailyhighs }\OperatorTok\StringTok{ }\KeywordTok{pivot_longer}\NormalTok{(}\DataTypeTok{cols =} \KeywordTok{starts_with}\NormalTok{(}\StringTok{"high"}\NormalTok{)) }\OperatorTok
\StringTok{  }\KeywordTok{mutate}\NormalTok{(}\DataTypeTok{date =} \KeywordTok{str_remove}\NormalTok{(name, }\StringTok{"high"}\NormalTok{) }\OperatorTok\StringTok{ }\NormalTok{lubridate}\OperatorTok{::}\KeywordTok{ymd}\NormalTok{(),}
         \DataTypeTok{kelvin =}\NormalTok{ (value }\OperatorTok{+}\StringTok{ }\FloatTok{273.15}\NormalTok{),}
         \DataTypeTok{Plot_ID =} \KeywordTok{tolower}\NormalTok{(Plot_ID),}
         \DataTypeTok{Treatment =} \KeywordTok{tolower}\NormalTok{(Treatment)) }\OperatorTok
\StringTok{  }\KeywordTok{select}\NormalTok{(date, Plot_ID, Treatment, kelvin) }\OperatorTok
\StringTok{  }\KeywordTok{rename}\NormalTok{(}\DataTypeTok{date.collection =}\NormalTok{ date, }\DataTypeTok{site.id =}\NormalTok{ Plot_ID, }\DataTypeTok{treatment.type =}\NormalTok{ Treatment)}
\end{Highlighting}
\end{Shaded}

Some of our LICOR measurements are takin in mV, some in micromoles/mol.
It shouldn't matter which, but we have to separate them, and calculate
their ppm of CO2 independently before merging them back together. We
then add in data about bucket size and temperature

\begin{Shaded}
\begin{Highlighting}[]
\NormalTok{mV =}\StringTok{ }\KeywordTok{filter}\NormalTok{(CO2_measurements, date.analysis }\OperatorTok\StringTok{ }\NormalTok{co2_standards_mV}\OperatorTok{$}\NormalTok{date)}
\NormalTok{mmol =}\StringTok{ }\KeywordTok{filter}\NormalTok{(CO2_measurements, date.analysis }\OperatorTok\StringTok{ }\NormalTok{co2_standards_mmol}\OperatorTok{$}\NormalTok{date)}


\NormalTok{mV_s =}\StringTok{ }\KeywordTok{merge}\NormalTok{(mV, standard_equations_mV, }\DataTypeTok{by.x =} \StringTok{"date.analysis"}\NormalTok{, }\DataTypeTok{by.y =} \StringTok{"date"}\NormalTok{)}
\NormalTok{mmol_s =}\StringTok{ }\KeywordTok{merge}\NormalTok{(mmol, standard_equations_mmol, }\DataTypeTok{by.x =} \StringTok{"date.analysis"}\NormalTok{, }\DataTypeTok{by.y =} \StringTok{"date"}\NormalTok{)}

\NormalTok{mV_s}\OperatorTok{$}\NormalTok{concentration =}\StringTok{ }\NormalTok{(mV_s}\OperatorTok{$}\NormalTok{slope }\OperatorTok{*}\StringTok{ }\NormalTok{mV_s}\OperatorTok{$}\NormalTok{peak.heights..mV.) }\OperatorTok{+}\StringTok{ }\NormalTok{mV_s}\OperatorTok{$}\NormalTok{intercept}
\NormalTok{mmol_s}\OperatorTok{$}\NormalTok{concentration =}\StringTok{ }\NormalTok{(mmol_s}\OperatorTok{$}\NormalTok{slope }\OperatorTok{*}\StringTok{ }\NormalTok{mmol_s}\OperatorTok{$}\NormalTok{peak.heights..um.m.) }\OperatorTok{+}\StringTok{ }\NormalTok{mmol_s}\OperatorTok{$}\NormalTok{intercept}

\NormalTok{all_concentrations =}\StringTok{ }\KeywordTok{bind_rows}\NormalTok{(mV_s, mmol_s)}

\NormalTok{all_concentrations =}\StringTok{ }\KeywordTok{merge}\NormalTok{(all_concentrations, buckets, }\DataTypeTok{by =} \StringTok{"bucket"}\NormalTok{)}
\NormalTok{all_concentrations}\OperatorTok{$}\NormalTok{date.collection =}\StringTok{ }\NormalTok{lubridate}\OperatorTok{::}\KeywordTok{mdy}\NormalTok{(all_concentrations}\OperatorTok{$}\NormalTok{date.collection)}
\NormalTok{all_concentrations}\OperatorTok{$}\NormalTok{pressure..atm. =}\StringTok{ }\DecValTok{1}

\NormalTok{all_concentrations =}\StringTok{ }\KeywordTok{merge}\NormalTok{(all_concentrations, high, }\DataTypeTok{by =} \KeywordTok{c}\NormalTok{(}\StringTok{"date.collection"}\NormalTok{, }\StringTok{"site.id"}\NormalTok{, }\StringTok{"treatment.type"}\NormalTok{))}
\end{Highlighting}
\end{Shaded}

Converting ppm CO2 back to grams CO2 using the ideal gas law

\begin{Shaded}
\begin{Highlighting}[]
\CommentTok{#calculate moles and grams of CO2}
\NormalTok{all_concentrations =}\StringTok{ }\NormalTok{all_concentrations }\OperatorTok\StringTok{ }\KeywordTok{mutate}\NormalTok{(}\DataTypeTok{moles =}\NormalTok{ ((pressure..atm. }\OperatorTok{*}\StringTok{ }\NormalTok{volume)}\OperatorTok{/}\NormalTok{(}\FloatTok{0.082057} \OperatorTok{*}\StringTok{ }\NormalTok{kelvin)) }\OperatorTok{*}\StringTok{ }\NormalTok{concentration}\OperatorTok{/}\DecValTok{1000000}\NormalTok{,}
                                                   \DataTypeTok{grams_co2 =}\NormalTok{ moles }\OperatorTok{*}\StringTok{ }\FloatTok{44.01}\NormalTok{)}
\end{Highlighting}
\end{Shaded}

Using our timepoints, our grams CO@ measurements, and the bucket size,
we can calculate the emission of CO2, in units of gCO2 * m\^{}-2 *
hr\^{}-1

\begin{Shaded}
\begin{Highlighting}[]
\CommentTok{#lay out data per bucket and summarize change}
\NormalTok{c_data =}\StringTok{ }\NormalTok{all_concentrations }\OperatorTok\StringTok{ }
\StringTok{  }\KeywordTok{select}\NormalTok{(date.collection, treatment.type, site.id, bucket, Time, grams_co2, QC.Y.N.) }\OperatorTok\StringTok{ }
\StringTok{  }\KeywordTok{filter}\NormalTok{(QC.Y.N. }\OperatorTok{!=}\StringTok{ "Y"}\NormalTok{) }\OperatorTok
\StringTok{  }\KeywordTok{mutate}\NormalTok{(}\DataTypeTok{timepoint =} \KeywordTok{round}\NormalTok{(Time}\OperatorTok{/}\DecValTok{60}\NormalTok{, }\DecValTok{0}\NormalTok{))}

\NormalTok{t0 =}\StringTok{ }\KeywordTok{filter}\NormalTok{(c_data, timepoint }\OperatorTok{==}\StringTok{ }\DecValTok{0}\NormalTok{)}
\NormalTok{t1 =}\StringTok{ }\KeywordTok{filter}\NormalTok{(c_data, timepoint }\OperatorTok{==}\StringTok{ }\DecValTok{1}\NormalTok{)}
\NormalTok{t2 =}\StringTok{ }\KeywordTok{filter}\NormalTok{(c_data, timepoint }\OperatorTok{==}\StringTok{ }\DecValTok{2}\NormalTok{)}

\NormalTok{h =}\StringTok{ }\KeywordTok{merge}\NormalTok{(t0, t1, }\DataTypeTok{by =} \KeywordTok{c}\NormalTok{(}\StringTok{"date.collection"}\NormalTok{, }\StringTok{"treatment.type"}\NormalTok{, }\StringTok{"site.id"}\NormalTok{, }\StringTok{"bucket"}\NormalTok{), }\DataTypeTok{all =}\NormalTok{ T)}
\NormalTok{h =}\StringTok{ }\KeywordTok{merge}\NormalTok{(h, t2, }\DataTypeTok{by =} \KeywordTok{c}\NormalTok{(}\StringTok{"date.collection"}\NormalTok{, }\StringTok{"treatment.type"}\NormalTok{, }\StringTok{"site.id"}\NormalTok{, }\StringTok{"bucket"}\NormalTok{), }\DataTypeTok{all =}\NormalTok{ T)}

\NormalTok{c_grams_time =}\StringTok{ }\NormalTok{h }\OperatorTok\StringTok{ }
\StringTok{  }\KeywordTok{select}\NormalTok{(date.collection, treatment.type, site.id, bucket, Time.x, grams_co2.x, Time.y, grams_co2.y, Time, grams_co2) }\OperatorTok\StringTok{ }
\StringTok{  }\KeywordTok{filter}\NormalTok{(site.id }\OperatorTok{!=}\StringTok{ "alt"}\NormalTok{) }\OperatorTok
\StringTok{  }\KeywordTok{drop_na}\NormalTok{() }\OperatorTok
\StringTok{  }\KeywordTok{rename}\NormalTok{(}\DataTypeTok{T0 =}\NormalTok{ Time.x, }\DataTypeTok{T1 =}\NormalTok{ Time.y, }\DataTypeTok{T2 =}\NormalTok{ Time, }\DataTypeTok{T0co2 =}\NormalTok{ grams_co2.x, }\DataTypeTok{T1co2 =}\NormalTok{ grams_co2.y, }\DataTypeTok{T2co2 =}\NormalTok{ grams_co2) }\OperatorTok
\StringTok{  }\KeywordTok{mutate}\NormalTok{(}\DataTypeTok{change1 =}\NormalTok{ ((T1co2 }\OperatorTok{-}\StringTok{ }\NormalTok{T0co2)}\OperatorTok{/}\NormalTok{((T1 }\OperatorTok{-}\StringTok{ }\NormalTok{T0)}\OperatorTok{/}\DecValTok{60}\NormalTok{))}\OperatorTok{/}\FloatTok{0.06334707}\NormalTok{,}
         \DataTypeTok{change2 =}\NormalTok{ ((T2co2 }\OperatorTok{-}\StringTok{ }\NormalTok{T1co2)}\OperatorTok{/}\NormalTok{((T2 }\OperatorTok{-}\StringTok{ }\NormalTok{T1)}\OperatorTok{/}\DecValTok{60}\NormalTok{))}\OperatorTok{/}\FloatTok{0.06334707}\NormalTok{,}
         \DataTypeTok{change_mean =}\NormalTok{ ((T2co2 }\OperatorTok{-}\StringTok{ }\NormalTok{T0co2)}\OperatorTok{/}\NormalTok{((T2 }\OperatorTok{-}\StringTok{ }\NormalTok{T0)}\OperatorTok{/}\DecValTok{60}\NormalTok{))}\OperatorTok{/}\FloatTok{0.06334707}\NormalTok{,}
         \DataTypeTok{date.collection =}\NormalTok{ lubridate}\OperatorTok{::}\KeywordTok{ymd}\NormalTok{(date.collection),}
         \DataTypeTok{changedif =}\NormalTok{ change1 }\OperatorTok{-}\StringTok{ }\NormalTok{change2) }\OperatorTok
\StringTok{  }\KeywordTok{pivot_longer}\NormalTok{(}\DataTypeTok{cols =} \KeywordTok{starts_with}\NormalTok{(}\StringTok{"change"}\NormalTok{))}

\NormalTok{means =}\StringTok{ }\NormalTok{c_grams_time }\OperatorTok\StringTok{ }\KeywordTok{filter}\NormalTok{(name }\OperatorTok{==}\StringTok{ "change_mean"}\NormalTok{)}
\NormalTok{single =}\StringTok{ }\NormalTok{c_grams_time }\OperatorTok\StringTok{ }\KeywordTok{filter}\NormalTok{(name }\OperatorTok{!=}\StringTok{ "change_mean"}\NormalTok{) }\OperatorTok\StringTok{ }\KeywordTok{filter}\NormalTok{(name }\OperatorTok{!=}\StringTok{ "changedif"}\NormalTok{)}
\NormalTok{difs =}\StringTok{ }\NormalTok{c_grams_time}\OperatorTok\StringTok{ }\KeywordTok{filter}\NormalTok{(name }\OperatorTok{==}\StringTok{ "changedif"}\NormalTok{)}
\end{Highlighting}
\end{Shaded}

A quick visualization of the distribution of our mean 2-hr emission
measurements

\begin{Shaded}
\begin{Highlighting}[]
\KeywordTok{ggplot}\NormalTok{(means, }\KeywordTok{aes}\NormalTok{(}\DataTypeTok{x =}\NormalTok{ value)) }\OperatorTok{+}
\StringTok{  }\KeywordTok{geom_histogram}\NormalTok{()}\OperatorTok{+}
\StringTok{  }\KeywordTok{xlab}\NormalTok{(}\KeywordTok{expression}\NormalTok{(}\StringTok{"g CO"}\NormalTok{[}\DecValTok{2}\NormalTok{]}\OperatorTok{*}\StringTok{" m"}\OperatorTok{^}\StringTok{"-2"} \OperatorTok{*}\StringTok{ "hr"}\OperatorTok{^}\StringTok{"-1"}\NormalTok{)) }\OperatorTok{+}
\StringTok{  }\KeywordTok{ylab}\NormalTok{(}\StringTok{"Count"}\NormalTok{)}
\end{Highlighting}
\end{Shaded}

\begin{verbatim}
## `stat_bin()` using `bins = 30`. Pick better value with `binwidth`.
\end{verbatim}

\includegraphics{co2_analysis_files/figure-latex/unnamed-chunk-8-1.pdf}
Distribution of individual single-hour measurements

\begin{Shaded}
\begin{Highlighting}[]
\KeywordTok{ggplot}\NormalTok{(single, }\KeywordTok{aes}\NormalTok{(}\DataTypeTok{x =}\NormalTok{ value)) }\OperatorTok{+}\StringTok{ }
\StringTok{  }\KeywordTok{geom_histogram}\NormalTok{() }\OperatorTok{+}
\StringTok{  }\KeywordTok{xlab}\NormalTok{(}\KeywordTok{expression}\NormalTok{(}\StringTok{"g CO"}\NormalTok{[}\DecValTok{2}\NormalTok{]}\OperatorTok{*}\StringTok{" m"}\OperatorTok{^}\StringTok{"-2"} \OperatorTok{*}\StringTok{ "hr"}\OperatorTok{^}\StringTok{"-1"}\NormalTok{)) }\OperatorTok{+}
\StringTok{  }\KeywordTok{ylab}\NormalTok{(}\StringTok{"Count"}\NormalTok{)}
\end{Highlighting}
\end{Shaded}

\begin{verbatim}
## `stat_bin()` using `bins = 30`. Pick better value with `binwidth`.
\end{verbatim}

\includegraphics{co2_analysis_files/figure-latex/unnamed-chunk-9-1.pdf}

Plotting the distribution of the difference between the first hour and
second hour. Values closer to zero indicate a constant rate of emission,
values further from zero indicate a change in rate during the
incubation.

\begin{Shaded}
\begin{Highlighting}[]
\KeywordTok{ggplot}\NormalTok{(difs, }\KeywordTok{aes}\NormalTok{(}\DataTypeTok{x =}\NormalTok{ value)) }\OperatorTok{+}
\StringTok{  }\KeywordTok{geom_histogram}\NormalTok{() }\OperatorTok{+}
\StringTok{  }\KeywordTok{xlab}\NormalTok{(}\KeywordTok{expression}\NormalTok{(}\StringTok{"g CO"}\NormalTok{[}\DecValTok{2}\NormalTok{]}\OperatorTok{*}\StringTok{" m"}\OperatorTok{^}\StringTok{"-2"} \OperatorTok{*}\StringTok{ "hr"}\OperatorTok{^}\StringTok{"-1"}\NormalTok{)) }\OperatorTok{+}
\StringTok{  }\KeywordTok{ylab}\NormalTok{(}\StringTok{"Count"}\NormalTok{)}
\end{Highlighting}
\end{Shaded}

\begin{verbatim}
## `stat_bin()` using `bins = 30`. Pick better value with `binwidth`.
\end{verbatim}

\includegraphics{co2_analysis_files/figure-latex/unnamed-chunk-10-1.pdf}

All of our calculated 2-hr mean emissions data across the last year,
separated by site, with treatments and control plots color-coded

\begin{Shaded}
\begin{Highlighting}[]
\KeywordTok{ggplot}\NormalTok{(means, }\KeywordTok{aes}\NormalTok{(}\DataTypeTok{x =} \KeywordTok{as.factor}\NormalTok{(date.collection), }\DataTypeTok{y =}\NormalTok{ value, }\DataTypeTok{fill =}\NormalTok{ treatment.type)) }\OperatorTok{+}
\StringTok{  }\KeywordTok{geom_boxplot}\NormalTok{(}\DataTypeTok{position =} \StringTok{"dodge"}\NormalTok{, }\DataTypeTok{width =} \DecValTok{1}\NormalTok{) }\OperatorTok{+}
\StringTok{  }\KeywordTok{facet_wrap}\NormalTok{(.}\OperatorTok{~}\NormalTok{site.id) }\OperatorTok{+}
\StringTok{  }\KeywordTok{theme_bw}\NormalTok{() }\OperatorTok{+}
\StringTok{  }\KeywordTok{theme}\NormalTok{(}\DataTypeTok{axis.text.x =} \KeywordTok{element_text}\NormalTok{(}\DataTypeTok{angle =} \DecValTok{90}\NormalTok{, }\DataTypeTok{hjust =} \DecValTok{1}\NormalTok{)) }\OperatorTok{+}
\StringTok{  }\KeywordTok{ylab}\NormalTok{(}\KeywordTok{expression}\NormalTok{(}\StringTok{"g CO"}\NormalTok{[}\DecValTok{2}\NormalTok{]}\OperatorTok{*}\StringTok{" m"}\OperatorTok{^}\StringTok{"-2"} \OperatorTok{*}\StringTok{ "hr"}\OperatorTok{^}\StringTok{"-1"}\NormalTok{)) }\OperatorTok{+}
\StringTok{  }\KeywordTok{xlab}\NormalTok{(}\StringTok{"Date"}\NormalTok{) }\OperatorTok{+}
\StringTok{  }\KeywordTok{scale_fill_brewer}\NormalTok{(}\DataTypeTok{name =} \StringTok{"Treatment"}\NormalTok{, }\DataTypeTok{palette =} \StringTok{"Paired"}\NormalTok{, }\DataTypeTok{labels =} \KeywordTok{c}\NormalTok{(}\StringTok{"Control"}\NormalTok{, }\StringTok{"Treatment"}\NormalTok{))}
\end{Highlighting}
\end{Shaded}

\includegraphics{co2_analysis_files/figure-latex/unnamed-chunk-11-1.pdf}


\end{document}
